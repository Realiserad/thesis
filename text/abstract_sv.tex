Uppsatsen undersöker hur en blockkedja kan användas för att bygga en decentraliserad publik nyckel-infrastruktur för webben. Vi ger ett designförslag på en blockkedja som drivs av en pålitlig grupp av noder, där en majoritet antas vara ärliga. Vårt huvudsakliga bidrag är utformningen av ett Proof of Stake-protokoll baserat på ett staketräd, vilket bygger på en idé som kallas follow-the-satoshi omnämnd i tidigare publikationer. 

Digitala identiteter sparas i ett autentiserat, självbalanserande träd som underhålls av noder anslutna till blockkedjenätverket. Våra preliminära beräkningar baserade på storleken av DNS-systemet visar att blockstorleken måste sättas till åtminstone 5.2 MB, medan varje nod med en månads transaktionshistorik måste spara ungefär 243 GB. Webbläsare och andra resurssnåla klienter måste synkronisera 13.6 MB data per år, och ladda ner ytterligare 3.7 KB för varje användarcertifikat som skall valideras.