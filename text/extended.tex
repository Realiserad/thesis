The binding between a physical identity and a public key is an important building block in computer security. This is typically done through digital documents following the X.509 standard, named X.509 certificates. These certificates are signed with the private key of a trusted intermediary called a certificate authority, who is responsible for checking the identity of the customer before issuing the certificate. Unfortunately, certificate authorities sometimes fail in their duty and issue fraudulent certificates. This violates the centralised trust model and makes it hard to trust any certificates issued by such certificate authority. One way of solving this problem is to adapt a decentralised model, where each user cross-signs their certificate with their private key. In such a system, a problem of consensus arises: How can we know which key is associated with a certain user? The emergence of the blockchain, originally used to store transactions for the Bitcoin cryptocurrency, offers a possibility to solve this problem without relying solely on a certificate authority. The aim of the thesis is to investigate how a blockchain can be used to build a decentralised public key infrastructure for the web, by proposing a custom federation blockchain which stores digital identities in an authenticated tree structure. Our main contribution is the design of a Proof of Stake protocol based on a stake tree which builds upon an idea called follow-the-satoshi used in previous papers. Our back-of-the-envelope calculations based on the size of the domain name system suggest a block size of at least 5.2 MB, while each blockchain node with a one-month transaction history need to store about 243 GB. Thin clients would have to synchronise about 13.6 MB of block headers per year, and download an additional 3.7 KB of proof data for every leaf certificate which is to be checked.